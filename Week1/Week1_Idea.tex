

\documentclass[10pt]{article}
\usepackage{amsmath, amssymb, geometry, graphicx, natbib, setspace}
\geometry{margin=0.95in}
\bibliographystyle{apalike}


\title{Week 1: Research Ideas Related to Fiscal Multipliers}
\author{Zorah Zafari}
\date{\today}

\begin{document}
\maketitle


\section*{Research Ideas}
Key Words: Fiscal Policy, Household Finance, Liquidity Constraints, Credit Constraints, Underbanked Households. 


\subsection*{1. Payday Lending and Regional Fiscal Stimulus: Local Multiplier Effects in High-Risk Credit Markets}
\begin{itemize}
\item \textbf{Motivation:} It is well established that there is significant heterogeneity in consumption responses to fiscal stimulus payments. For example, \citet{bakeretall} find that the effects of stimulus are much greater when targeted at households with a low level of liquidity. Individuals with less than \$100 in their accounts spend over 40 percent of stimulus within first month. Moreover, using a large-scale survey of US consumers, \citet{Coinbon} study how transfers from CARES act affect individuals' consumption, savings, and labor supply decisions. They find that relative to white households, black households save less and pay off more debt, on average. Liquidity constrained individuals use less of stimulus to save and pay more debt relative to unconstrained households. They also have higher MPCs than non-liquidity constrained households. To better understand the heterogeneity in MPCs arising from liquidity constraints combined with my interests in payday loans, I am motivated to investigate how high reliance on payday lending may influence regional responses to fiscal expansions due to immediate liquidity needs. Particularly, do regions with high reliance on payday lending and other subprime credit markets respond differently to fiscal expansions? (Potentially due to immediate liquidity needs and credit constraints.)
\item \textbf{Research Question:} Does the presence of payday lending in a region affect the local fiscal multiplier? Are local multipliers larger or smaller in communities with high subprime credit dependence? Does access to payday loans amplify or suppress consumption responses to fiscal transfers? How can we aggregate these regional findings to national estimates? 
\item \textbf{Empirical Strategy:} Map payday lending locations and measure the density of payday loan shops per capita. Match these geographic measures with ARRA spending data or COVID relief disbursements at the county level. Use instrumental variables (IV) based on historical credit access and state-level payday lending regulations to identify exogenous variation in payday lending exposure. Or use physical distance to payday loan bank as exogenous variation? Estimate local multipliers using regional consumption and income responses.
\item \textbf{Connection to Literature:} Extends \citet{nakamura2011fiscal} on regional multipliers by introducing local credit constraints and integrates \citet{pinardon2023crowding} concept of financial crowding out, but applied to subprime credit markets.
\end{itemize}


\subsection*{2. Financial Inclusion and the Transmission of Fiscal Multipliers}
\begin{itemize}
\item \textbf{Motivation:} According to data from the Survey of Consumer Finance (SCF), there are notable gaps in access to financial markets espically among low-income, Black and Hispanic adults. Six percent of all adults were unbanked in 2023 but among low-income adults, it is twenty-three percent. Regions with higher unbanked populations or reliance on alternative financial services (e.g., check cashing, title loans) may experience different consumption responses to fiscal injections due to limited banking infrastructure.
\item \textbf{Research Question:}  Does the degree of financial inclusion in a region influence the effectiveness of fiscal stimulus on local consumption? Are regions with high rates of unbanked households less responsive to fiscal stimulus? Does direct deposit versus check disbursement affect the velocity of money in low-credit regions?
\item \textbf{Empirical Strategy:} Map financial inclusion metrics (e.g., FDIC surveys, CFPB data, or SCF data) to regional ARRA spending and stimulus payments. Maybe analyze regions with state policies on check cashing limits or direct deposit mandates or try to measure consumption spillovers in regions with different banking access levels.

\item \textbf{Connection to Literature:} Builds on \citet{woodford2011simple} and \citet{pinardon2023crowding} by testing how financial frictions affect fiscal transmission and crowding out from limited financial infastructure.
\end{itemize}




\newpage 

\bibliography{week1bib}
\end{document}