
\documentclass[10pt]{article}
\usepackage{amsmath, amssymb, geometry, graphicx, natbib, setspace}
\geometry{margin=0.95in}
\bibliographystyle{apalike}


\title{Week 2: Much Ado about Phillips Curves}
\author{Zorah Zafari}
\date{\today}

\begin{document}
\maketitle

\section*{Research Ideas}



\subsection*{1. Inflation Expectations and Regional Phillips Curves}

\begin{itemize}
\item \textbf{Motivation:}
\begin{itemize}
\item The Federal Reserve's dual mandate aims to promote maximum employment and stable prices. Central to achieving these goals is the Fed's ability to manage inflation expectations. After the March FOMC meeting earlier this year, Federal Reserve Chair Jerome Powell emphasized that ``anchored inflation expectations'' are at the heart of the Fed's policy framework. This concern extends beyond financial markets to households, as underscored by Coibion et al. (2022), who highlight that household inflation expectations significantly influence firms' pricing decisions and household consumption-savings behavior. These expectations are often measured through survey data, such as the Michigan Survey of Consumers or NY Fed Survey Data.

\item The Phillips Curve, a well-established economic concept, represents the inverse relationship between inflation and unemployment. McLeay and Tenreyro (2020) discuss the identification challenges associated with the Phillips Curve and provide evidence of a steeper Phillips Curve in U.S. regional data, despite reduced-form evidence suggesting a flattening trend. However, while McLeay and Tenreyro (2020) argue that optimal monetary policy can flatten the Phillips Curve by stabilizing inflation and output, their analysis does not explicitly focus on the stabilization of long-term inflation expectations as the primary mechanism. Hazell et al. (2020) extend this insight to regional Phillips Curves but do not directly test the role of household expectations. Coibion et al. (2022) emphasize that household inflation expectations are heterogeneous, shaped by personal experience, and significantly influence consumption decisions.

\item Furthermore, Bracha and Tang (2023) document that attention to inflation increases when inflation levels are high. They introduce two novel measures of inflation inattention: (1) the share of respondents who say ``Don't know'' when asked for an inflation estimate, and (2) the estimation errors of inflation expectations. Their analysis confirms that when inflation is high, consumers pay more attention to price changes, which they connect to the cost-benefit trade-off of acquiring information about inflation. 

\item My goal is to try to explicitly link the flattening of the Phillips curve not just to anchored long-term expectations, but also to the degree of consumer attention to inflation. According to Bracha and Tang (2023), when inflation is low, households pay less attention to price changes, which could lead to more dispersed expectations and a muted response in wage and price setting. Conversely, when inflation is high, attention rises, potentially steepening the Phillips Curve in high-volatility regions. 
\end{itemize}

\item \textbf{Research Questions:}
\begin{enumerate}
\item \textbf{To what extent do long-term inflation expectations explain the flattening of the Phillips Curve across different U.S. states? Does the anchoring of expectations differ between states with high economic volatility and those with stable growth?}
\begin{itemize}
\item States with historically high economic volatility (e.g., energy-dependent regions like Texas) have less-anchored inflation expectations, resulting in steeper Phillips Curves.
\item States with stable economic growth histories (e.g., California, Massachusetts) exhibit stronger expectation anchoring and flatter Phillips Curves.
\end{itemize}
\item \textbf{How does political alignment with the federal administration influence inflation expectations and Phillips Curve dynamics?}
\begin{itemize}
    \item Political alignment with the federal administration may affect expectation anchoring due to differences in economic policies and fiscal stability.
\end{itemize}
\item \textbf{What role do regime shifts play in altering household inflation expectations across states?}
\begin{itemize}
    \item Regime shifts, particularly between inflation-sensitive and inflation-tolerant administrations, may cause significant adjustments in household inflation expectations, impacting regional Phillips Curves.
\end{itemize}
\end{enumerate}



\item \textbf{Actionable Tasks and Research Plan:}

\begin{itemize}
\item{Data Collection}
\begin{itemize}
    \item Survey of Professional Forecasters (SPF) for inflation expectations.
    \item Michigan Survey of Consumers: Household-level inflation expectations.
    \item State-level CPI Data: Inflation rates by state.
    \item State-level Unemployment Data (LAUS) or CPS microdata.
    \item Political Alignment Data: Historical voting patterns and federal administration alignment. These I think are readily available or can probably be scraped online. 
\end{itemize}

\item{Methodology}
\begin{itemize}
    \item Try to estimate State-specific Phillips Curves with and without long-term inflation expectations starting with a very simple model where the LHS includes $\pi_{st}$ (inflation for state $s$ in time $t$) and the right hand side might include a measure of state unemployment $U_{st}$, expected inflation for the next period $E[\pi_{st+1}]$, attention level to inflation in state $s$ at time $t$, and some state and time fixed effects. A second iteration might include interaction terms for states with historically high GDP volatility and political alignment to test if their expectations differ systematically. 
\end{itemize}

\item{Research Design and Identification Strategy}
\begin{itemize}
    \item Use historical GDP volatility as an instrument for attention to inflation.
    \item Control for political realignment and test the robustness using low-volatility states as a placebo.
    \item Use lagged political alignment to address potential endogeneity.
\end{itemize}

\end{itemize}




\item \textbf{Conclusion}
By incorporating attention mechanisms and political alignment, I propose to bridge existing theoretical gaps and offer policy-relevant findings. For example, results from this sort of analysis could inform more tailored and region-specific monetary policies. Regional stabilization policies could target expectation anchoring through direct communication and enhanced local monetary engagement. Additionally, the Fed could benefit from incorporating regional volatility measures and political alignment into its forward guidance communication to better stabilize inflation expectations.
\end{itemize}



\subsection*{2. Intrastate Phillips Curve: Alternative Heterogeneity?}
The existing literature predominantly examines the Phillips Curve at the state or regional level, often aggregating across diverse populations. However, this regional aggregation may mask significant heterogeneity in inflation-unemployment trade-offs across demographic groups within states. Building on Herreno and Pedemonte (2022) work on hand-to-mouth consumers and the distributional impacts of monetary policy, I propose investigating the slope of the Phillips Curve across demographic lines such as age, race, education, and income levels. I will add that I have not thought too deeply what added information demographic groups may add that isnt alreayd being explained by income (this section really is just random thoughts that were coming to my head as I was reading the paper). 

I am just thinking that labor market flexibility and wage sensitivity may differ substantially across demographic segments, which may contribute to variations in the Phillips Curve's slope within states. For instance, younger workers may face more wage volatility and weaker bargaining power, leading to a steeper Phillips Curve. Historical labor market frictions and varying degrees of employment protections might result in demographic-specific inflation sensitivities. Higher-educated populations may have more job stability, leading to a flatter Phillips Curve relative to less-educated cohorts. These can be tested empirically with first-stage reduced-form analysis using Current Population Survey (CPS) microdata to estimate group-specific Phillips Curves within states. One potential limitation is the sufficiency of sample sizes for subgroup analysis within smaller states; however, pooling across years or using grouped estimators could mitigate this concern.

\subsection*{3. Sectoral Phillips Curves in the Gig Economy}

Another random thought: rapid expansion of the gig economy introduces new labor market dynamics that may fundamentally alter the traditional Phillips Curve relationship. Workers can enter and exit the gig economy with minimal frictions, reducing the wage pressure traditionally associated with low unemployment. Another extension of the Herreno and Pedemonte (2022) analysis might be creating a TANK model where locations differ by the size of the gig economy?



\end{document}


