\documentclass[10pt]{article}
\usepackage{amsmath, amssymb, geometry, graphicx, natbib, setspace}
\geometry{margin=0.95in}
\bibliographystyle{apalike}


\title{Week 4: Macroeconomics of Consumption with Heterogeneity}
\author{Zorah Zafari}
\date{\today}

\begin{document}
\maketitle

\section*{Research Ideas}
\subsection*{1. Redlining and the Unequal Pass-Through of Monetary Policy}

\begin{itemize}
\item \textbf{Motivation:}
\begin{itemize}
\item Although we did not discuss the Berger et al. (2021) paper in class, I found it quite interesting. The paper shows that the effectiveness of monetary policy through the mortgage refinancing channel depends on the distribution of mortgage rate gaps. Only households with positive rate gaps $frac>0$ respond strongly to rate cuts, making monetary transmission path-dependent.

\item The authors argue that refinancing constraints from lender's side due to high LTV or low credit scores are not the main driver of observed patterns. Instead, monetary policy pass-through depends on the current composition of mortgage contracts, which in turn is shaped by past refinancing opportunities.

\item I propose that the composition of mortgage contracts—and therefore the distribution of rate gaps—is shaped by the long history of racial exclusion in U.S. housing and credit markets. Specifically, redlining and discriminatory mortgage policies shaped where and when households were able to enter the mortgage market, which lenders they used, and at what terms. Even if redlining occurred decades ago, its legacy may still show up in today's mortgage stock: households in formerly redlined areas may have entered the mortgage market later, with worse terms, and may have refinanced less frequently in past low-rate periods. As a result, they may systematically hold fewer loans with high refinancing potential today, leading to a lower $frac>0$.

\item This project would evaluate whether the historical geography of credit exclusion helps explain variation in the potency of monetary policy today—linking structural inequality to the transmission mechanism itself.
\end{itemize}

\item \textbf{Research Questions:}
\begin{enumerate}
    \item Do neighborhoods with a redlining history exhibit persistently lower $frac > 0$ values, even conditional on income and credit score?
    \item Are households in formerly redlined areas less likely to have benefited from prior refinancing waves, leaving them with less refinancing capacity during future rate cuts?
    \item Does the legacy of redlining reduce monetary policy pass-through via the mortgage channel- e.g., credit expansion following a rate cut or change in durable consumption?
    \item Can we interpret redlining as a slow-moving exposure that shapes the mortgage composition today, thereby producing persistent heterogeneity in monetary policy effectiveness across space and race?
\end{enumerate}

\item \textbf{Empirical Strategy:}
\begin{itemize}
    \item \textbf{Data Sources:}
    \begin{itemize}
        \item Home Mortgage Disclosure Act (HMDA) data for loan originations, borrower race/income, and refinancing activity by tract and year.
        \item Historical HOLC redlining maps (Mapping Inequality Project) geocoded to census tracts.
        \item Potentially use CoreLogic for Rate gap or $frac > 0$ since the department just purchased this dataset.
    \end{itemize}
    \item \textbf{Approach:}
    \begin{itemize}
        \item Construct a panel of census tracts with HOLC redlining grades and link them to current mortgage activity and refinancing behavior.
        \item Regress tract-level $frac > 0$ on historical redlining indicators, controlling for current income, home values, and borrower credit scores.
        \item Estimate difference-in-differences or event study models around monetary policy shocks (e.g., Romer \& Romer shock dates) to examine whether tracts with redlining histories exhibit muted durable consumption or credit responses relative to never-redlined tracts.
        \item Explore heterogeneity by race within tracts using borrower-level HMDA data.
    \end{itemize}
\end{itemize}
\item \textbf{Conclusion:}
By embedding the history of redlining into the Berger et al. (2021) framework, I aim to assess the distributional equity of monetary policy pass-through. This work could help policymakers better understand how structural inequality shapes macroeconomic effectiveness, and motivate targeted interventions in the mortgage market to repair historic exclusion.
\end{itemize}





\subsection*{2. Empirically Testing Sticky Expectations in HANK Models}

\begin{itemize}
\item \textbf{Motivation:}
\begin{itemize}
    \item Auclert, Rognlie, and Straub (2020) estimate a Heterogeneous-Agent New Keynesian (HANK) model that can match both ``micro jumps'' in household consumption and ``macro humps'' in aggregate responses to monetary shocks by incorporating sticky expectations.
    \item Sticky expectations dampen households’ intertemporal substitution response to interest rate changes, generating gradual consumption responses (macro humps) rather than sharp ones (macro jumps).
    \item However, sticky expectations in their model are assumed rather than tested directly against micro data. Household survey data offer an opportunity to test this feature empirically and since I am already working with such survey data (: maybe we can empirically test ``sticky expectations''?
\end{itemize}

\item \textbf{Research Questions:}
\begin{itemize}
    \item Do household inflation expectations exhibit stickiness consistent with slow updating, especially following monetary policy shocks? Can we identify state-dependent vs. time-dependent updating patterns using real-time forecast errors in survey data?
\end{itemize}

\item \textbf{Actionable Tasks and Research Plan:}
\begin{itemize}
    \item \textbf{Data Collection:}
    \begin{itemize}
        \item University of Michigan Survey of Consumers (cross-sectional), used for attention proxies (e.g., “don’t know” responses).
        \item FOMC statement release dates and Romer \& Romer monetary policy shocks to identify surprise movements.
    \end{itemize}
    \item \textbf{Methodology:}
    \begin{itemize}
        \item Estimate distributed lag models of inflation expectation updating around identified monetary shocks. Then compute persistence of forecast errors to quantify stickiness. Stratify by income, education, and financial literacy to test heterogeneity in attention and updating behavior.
         \item Try to calibrate how often households update their expectations (like in Calvo or Carroll models) by seeing if the patterns in the survey data—how slowly people revise their forecasts after monetary shocks—line up with what a sticky expectations model would predict, similar to what Auclert, Rognlie, and Straub (2020) do.
    \end{itemize}
\end{itemize}

\item \textbf{Conclusion:}
With this empirical analysis, I aim to bridge the micro and macro features of sticky expectations by directly estimating the speed and heterogeneity of household updating behavior. It will empirically discipline a key parameter in the Auclert et al. (2020) model, improve the microfoundations of HANK modeling, and contribute to the design of forward guidance communication strategies.

\end{itemize}



\end{document}

